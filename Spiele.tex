\documentclass[a4paper]{report}
\usepackage[ngerman]{babel} 
\usepackage[utf8]{inputenc}
\usepackage[T1]{fontenc}
\usepackage[a4paper]{geometry} 
\geometry{
    a4paper,
    total={180mm,270mm},
    left=15mm,
    top=15mm,}
\usepackage[pdfpagemode=FullScreen]{hyperref}
\pagenumbering{gobble}

\begin{document}

\begin{center}%%%%%%%%%%%%%%%%%%%%%%%%%%%%%%%%%%%%%%%%
    \huge Wochenspannung
\end{center}

\section*{Vampirspiel}
Zu Beginn des Ausflugs werden Karten/Zettel an alle Spieler verdeckt ausgeteilt. Auf einer Karte steht ,,Vampir'' auf den anderen Karten ,,Bürger''. Jeder Vampir hat die Aufgabe, alle anderen Mitspieler zu Vampiren zu verwandeln. Ein Vampir kann einen Bürger in einen Vampir verwandeln, wenn er mit der Person allein ist, sie antippt und ,,Vampir'' sagt. Die Bürger können Sitzungen einberufen und abstimmen, ob ein Mitspieler als Vampir verdächtigt wird. Wird ein Spieler mit der Mehrheit der Stimmen für einen Vampir gehalten, scheidet er aus dem Spiel aus. Ziel der Bürger ist es, alle Vampire aus dem Spiel zu werfen. Sind am Ende nur Bürger übrig, haben die Bürger gewonnen. Sind alle Spieler zu Vampiren geworden, haben die Vampire gewonnen.

\section*{Verbotene Worte}
Zu Beginn des Ausflugs wird von allen Teilnehmern eine Liste erstellt mit Worten, die während der gesamten Zeit verboten sind. Diese Worte dürfen während des Ausflugs nicht verwendet werden. Fall ein Teilnehmer eines dieser Worte doch verwendet, verliert er und scheidet aus dem Spiel aus. Gewonnen hat, wer am Ende als letztes übrig bleibt.

\section*{Bingo}
Der Unterschied zur normalen Bingo Version sind die Begriffe. Anstatt wie typisch Zahlen zu nutzen, werden Begriffe, die mit dem Ausflug zu tun haben gezogen. Dies bedarf schon einer gewissen Vorbereitung im Vorfeld der Reise, denn die Bingokarten und Zettel müssen erst zuhause gestaltet oder im Internet heruntergeladen werden.

\noindent\makebox[\linewidth]{\rule{\paperwidth}{0.4pt}}
\begin{center}%%%%%%%%%%%%%%%%%%%%%%%%%%%%%%%%%%%%%%%%
    \huge Knobelspaß
\end{center}

\section*{Silent Karaoke}
Jeder Spieler wählt einen Song aus und versucht diesen nur durch Gestik den anderen Spielern zu vermitteln. Es darf dabei nicht gesprochen, gesungen oder anderweitig Geräusche gemacht werden. Der erste Spieler der den Song richtig erräht hat gewonnen.

\section*{Wer bin ich}
Alles was du brauchst, ist ein Zettel, einen Stift und eventuell Klebeband.
Jeder schreibt dem jeweils anderen einen Namen von einer bekannten Persönlichkeit auf den Zettel und klebt es ihm auf die Stirn, ohne, dass derjenige sieht, was darauf steht. Jetzt muss jeder anhand Fragen reihum herausbekommen, wer er ist. Dabei dürfen die Fragen nur mit Ja oder Nein beantwortet werden. Derjenige, der als Erstes herausfindet, wer er ist, hat gewonnen. Das Spiel kann auch mit Tieren oder Berufen gespielt werden.

\section*{Stadt, Land, Fluss}
Du brauchst lediglich ein paar Blätter und Stifte. Nun legt ihr gemeinsam die Kategorien von Stadt, Land, Fluss bis über Berge, Tiere, Pflanzen, Berufe usw. fest. Dabei kannst du deiner Kreativität freien Lauf lassen und dir Kategorien wie Fußballvereine, VIPs, Küchengeräte, Cocktails oder Filmhelden ausdenken.
Der erste Spieler geht leise von A beginnend das ABC durch. Der Spieler rechts sagt irgendwann Stopp. Der Buchstabe, bei dem gestoppt wurde, wird nun als Grundlage für die Partie genommen. Wurde beispielsweise bei ,,H'' gestoppt, müssen alle Kategorien mit dem Anfangsbuchstaben ,,H'' ausgefüllt werden. Der Spieler, der seine Zeile als erstes vollständig ausgefüllt hat, stoppt die Runde. Daraufhin werden alle Ergebnisse vorgelesen und die Punkte verteilt.
Ist die Kategorie leer, gibt es 0 Punkte. Hat niemand anderes in der Kategorie das gleiche Wort gibt es drei Punkte, hat ein anderer Spieler das gleiche Wort gefunden gibt es 2 Punkte, haben mehrere Spieler das gleiche Wort gibt es einen Punkt.

\section*{Ich packe meinen Koffer}
Jeder Mitspieler nennt einen Gegenstand, den er ,,in seinen Koffer packt''. Der nächste Mitspieler zählt alle bereits genannten Gegenstände in der richtigen Reihenfolge auf und fügt einen weiteren Begriff hinzu. Die Liste wird so immer länger und wer als Erstes einen Gegenstand vergisst, scheidet aus.

\section*{Tabu}
Es werden ein oder mehrere verbotene Worte ausgewählt. Zwei Spieler als Moderator und als Befragter treten gegeneinander an. Der Moderator stellt dem Befragten viele Fragen. Dieser darf aber nicht mit Ja oder Nein antworten. Wenn der Befragte doch ausversehen mal das verbotene Wort ausspricht, ist die Runde vorbei und die Spieler tauschen die Rollen.

\section*{Schreib-meine-Geschichte-weiter-Spiel}
Für das Spiel brauchst du lediglich ein Blatt Papier und einen Stift.
Der Spieler, der anfängt, schreibt zwei Sätze untereinander. Den ersten Satz knickt er um und gibt den Zettel dem nächsten Spieler. Jetzt kann der zweite Spieler nur den zweiten, unteren Satz lesen. Der zweite Spieler überlegt sich nun, was zum Satz seines Vorgängers passen könnte und schreibt jetzt seine eigenen zwei Sätze darunter. Er knickt das Blatt wieder so um, dass man nur noch seinen letzten Satz lesen kann und gibt den Zettel an den nächsten weiter. Heraus kommt meistens eine haarsträubende und sehr lustige Geschichte.

\section*{Memory}
Falls ihr kein Memory dabei habt, könnt ihr euch mit Pappe, Stifte und Schere auch ein Memory-Spiel selbst basteln.
Es gibt jeweils zwei Karten, die identisch sind. Alle Karten werden vermischt und verdeckt ausgelegt. Der Reihe nach kann jeder Spieler zwei Karten aufdecken. Sind zwei Karten identisch, gehen beide an den Spieler, geben diesem einen Punkt und er darf erneut zwei Karten aufdecken. Sind zwei aufgedeckte Karten nicht identisch, werden diese wieder verdeckt und der nächste Spieler ist am Zug.

\section*{Tabu}
Bei Tabu bilden die Spieler zwei Teams. Reihum darf jeder Spieler Begriffe erklären und sein eigenes Team muss den Begriff erraten. Dabei dürfen bestimmte Begriffe, die auf den Karten sind, nicht genannt werden. Jeder Spieler hat durch die Sanduhr (oder Countdown auf dem Handy) nur eine bestimmte Zeit, um seine Begriffe zu erklären.

\section*{Montagsmaler}
Bilden Sie zwei Gruppen, die gegeneinander im Wettkampf spielen. Gruppe A fängt an und der erste Spieler bekommt von Ihnen einen Zettel auf dem ein Begriff steht. Diesen Begriff gilt es nun so schnell wie möglich zu malen, so dass seine Gruppe ihn erraten kann. Wenn Sie einen Tageslichtschreiber hätten, wäre dieses Spiel natürlich am wirkungsvollsten.

Wurde der Begriff erraten, kommt der nächste aus seiner Mannschaft mit malen dran. Das geht solange, bis zwei Minuten um sind. Wer in der vorgegebenen Zeit die meisten Begriffe erraten hat, gewinnt das Spiel. Es können auch statt Begriffen Sprichwörter, Pop-Gruppen-Namen, Filmtitel, Orte, Länder oder Tätigkeiten sein. Die zweite Variante wäre, statt die Begriffe zu malen, sie pantomimisch darzustellen.

\section*{Mord in Palermo}
Mindestens 8 Leute sollten mitspielen. Besonders gemütlich ist es gemeinsam am Lagerfeuer, denn hier ist die gewisse Krimi Atmosphäre inklusive.

Der Spielleiter verteilt Zetteln auf denen B (Bürger), (2x oder 1 mal) M (Mörder) und P (Polizist) steht. Natürlich ist geheim, wer welchen Zettel bekommen hat. Nun macht der Spielleiter reihum Aussagen, welche die anderen befolgen.

\begin{enumerate}
    \item ,,Es ist Nacht auf unserem Campingplatz. Schließt die Augen''
    \item ,,Die Mörder wachen auf.'' Die Mörder öffnen ihre Augen und sehen wer ihr Partner im Verbrechen ist. Anschließend wählen diese ihr Opfer. Aber Vorsichtig, die andere Spieler sollen nichts mitbekommen!
    \item ,,Die Mörder schlafen wieder ein und der/die Polizist*in wacht auf.'' Der/die Polizist*in versucht nun einen der Mörder auszumachen. Der Spielleiter darf nun den/die Polizist*in unbemerkt von den anderen mitteilen, ob er/sie recht hat.
    \item ,,Es ist wieder Tag auf unserem Campingplatz. Alle wachen auf. Und oh weh,..die (Name derjenigen Person, die von den Mördern ausgewählt wurde) ist ermordet worden''
    \item Nun darf die ganze Gruppe rätseln. Wer ist der Mörder? Auch die Mörder tuen unschuldig und verdächtigen andere. Das Knifflige: Der/die Polizist*in, der/die eventuell schon weiß, darf nicht sagen, dass er/ sie der/die Polizist*in ist, versucht aber die anderen von seiner/ ihrer Theorie zu überzeugen.
    \item Über die von den Mitspielern nominierten Mörder wird per Abstimmung entschieden, ob sie im Spiel bleiben oder nicht. Die Person mit den meisten Stimmen scheidet aus, verrät aber ihre Karte nicht.
    \item Nun beginnt der Spielleiter wieder von vorne und läutet so die zweite Runde ein.
\end{enumerate}

Das Spiel endet, sobald alle Mörder ausgeschieden sind oder sobald die Mörder alle Bürger und Polizisten ermordet haben. Wenn alle Mörder ausgeschieden sind, verkündet der Spielleiter: Es ist Tag und keiner wurde ermordet! Das Spiel kann super variiert und an die Gruppe angepasst werden: welche Tat begangen wurde, welche Geschichte der Spielleiter konstruiert usw.

\section*{Karten- und Brettspiele}
Nicht die einfallsreichste Idee für ein Campingspiel aber eine der beliebtesten sind Karten- und Brettspiele aller Art.

Beispiele: Quartett, Uno, Mensch-Ärgere-Dich-Nicht, Mau-Mau, Rommé, Ski Jo, Skip Bo, Skat, Offiziersskat

\section*{Schattenspiele}
Wenn Sie abends im Dunkeln im Zelt ein Licht anmachen, können Sie mit Ihren Händen und Fingern Schatten auf die Zeltwand werfen. Spielen Sie eine Geschichte nach oder veranstalten Sie einen Wettbewerb, um zu sehen, wer die meisten Figuren formen kann.

\section*{Stille Post}
Am besten ist es einen Stuhlkreis zu machen. Der Beginner legt einen Begriff, eine Aussage oder einen sinnvollen Satz zurecht und flüstert diesen seinem Tischnachbarn leise ins Ohr, damit ihn kein anderer mitbekommt. Der zweite Spieler flüstert das Verstandene nun in das Ohr seines nächsten Nachbarn. So geht es reihum weiter. Ist die Stille Post beim letzten Spieler angekommen, darf dieser laut in der Gruppe vortragen, was er verstanden hat.

\noindent\makebox[\linewidth]{\rule{\paperwidth}{0.4pt}}
\begin{center}%%%%%%%%%%%%%%%%%%%%%%%%%%%%%%%%%%%%%%%%
    \huge Aktiv
\end{center}

\section*{Simon Says}
Ein Spieler ist der besagte Simon. Dieser denkt sich Aktivitäten und Tätigkeiten aus, die er entweder mit oder ohne ,,Simon sagt''.  Nur bei der Anweisung mit der Phrase sollen die Mitspieler die Aktion ausführen. Wer sie bei einer normalen Aussage ohne ,,Simon sagt'' dennoch macht, bekommt einen Minuspunkt. Wann das Spiel endet, entscheidet ihr. Und damit keiner zu kurz kommt ist bei der nächsten Runde einfach ein anderer der Simon oder die Simone.

\section*{Bierpong}
Vor Beginn des Spiels sind die Bier Pong Becher auf den Bier Pong Tisch aufzustellen. Auf jeder Seite stehen dafür 10 Becher in Form einer Pyramide. Die Spitze der Pyramide zeigt dabei in die Mitte des Spieltisches. Die hinterste Reihe sollte so aufgestellt sein, dass diese maximal ein bis zwei Zentimeter von der Tischkante wegsteht. Wenn die Becher im Laufe des Spiels verrutschen, sind diese erneut in die richtige Aufstellung zurückzustellen.

Beide Spieler beziehungsweise Teams, haben die Möglichkeit, zwei Re-Racks zu spielen. Re-Racks ermöglichen den Spielern, die Becher des Gegners nach Belieben zu verschieben. Der sinnvolle Einsatz der Re-Racks ist, wenn das Spielfeld des Gegners löchrig wird. Re-Racks sind für beide Spieler dann erlaubt, wenn der Gegner noch drei, vier oder sechs verbleibende Becher hat. Der letzte Becher ist grundsätzlich mittig an der Tischkante des Bier Pong Tisches zu stellen.

Wichtig zu wissen ist jedoch, dass der Ball nur berührt werden darf, wenn dieser zuvor den Tisch oder den Cup berührt hat. Berührt der Spieler den Ball vorher, dann erfolgt eine Penalty-Strafe.

Beim Bounce Shot muss das gegnerische Team gleich zwei Becher trinken. Der Ball springt zunächst auf dem Spieltisch auf, bevor dieser in den Becher geht. Hat der Ball den Tisch berührt, hat das gegnerische Team noch die Chance zu verteidigen und den Ball wegzuschlagen.

Fällt im Verlauf des Spiels ein Becher um, dann ist dieser wieder aufzustellen, außer der Ball war bereits im Cup. Während des Spiels sind Ablenkungsmanöver des Gegners erlaubt. Nicht erlaubt ist es allerdings, Becher, Tisch oder den Gegner selbst zu berühren. Ebenso verboten ist das Erzeugen eines Windstoßes, der den Ball beeinflussen würde.

\section*{Flunkyball}
Es gibt zwei Teams die üblicherweise aus jeweils 1-6 Spielern bestehen. Diese stehen sich gegenüber und in der Mitte steht eine mit einer Flüssigkeit befüllten Flasche. Jeder Spieler hat ein Getränk vor sich stehen, welches möglichst schnell getrunken werden muss (normalerweise eine 0,5l-Bierdose).

Die Teams werfen abwechselnd mit einem Ball auf die Flasche in der Mitte. Nur wenn diese umfällt, darf solange getrunken werden, bis das gegnerische Team sie wieder aufgestellt hat und wieder in Reihe steht.

\section*{Tauziehen}
Acht Spieler stehen sich gegenüber und dazwischen liegt ein mindestens 35 Meter langes Seil. Von der Mitte des Seils aus befindet sich nach vier Metern zu beiden Seiten eine weiße Markierung. Das Team, welches es schafft, das andere Team über die Markierung zu ziehen, hat gewonnen. Damit ein Sieg feststeht, müssen zwei von drei Zügen gezogen werden.

Es ist verboten sich hinzusetzen, Handschuhtragen ist nicht erlaubt und das Seil wird unter dem Oberarm durchgeführt.

\section*{Komm mit lauf weg}
Alle Spieler stellen sich im Kreis auf. Einer ist der ,,Fänger''. Er geht außen am Kreis herum, solange er möchte, und tippt irgendwann einem Mitspieler auf die Schulter und sagt entweder ,,komm mit'' oder ,,Lauf weg''.

Beim Signal ,,komm mit'' rennt der angetippte Spieler in dieselbe Richtung, in die auch der ,,Fänger'' rennt.
Beim Signal ,,Lauf weg'' muss der aktivierte Spieler in die entgegengesetzte Richtung rennen.

Wer zuerst an der freigewordenen Lücke angekommen ist, darf dortbleiben und der andere ist der neue/alte Fänger.

\section*{Schnitzeljagd}
Der Inhalt der Schnitzeljagd kann alles mögliche sein. Beispielsweise das Sammeln von Dingen in der Natur (Blättern, Stöcke bis hin zu schwerer zu findenden Gegenständen wie Insekten oder Blumen). Dabei können entweder Teams spielen oder jeder spielt für sich. Dieses Spiel kann sowohl gut auf dem Campinggelände als auch auf einer Wanderung oder einem Spaziergang gespielt werden. Planung ist erforderlich.

Am besten gestaltet ihr Zettel, die die zu findenden Dinge abbilden, sodass sie, sobald sie gefunden wurden, auch abgehakt werden können oder gegebenenfalls weitere Notizen Platz finden.

\section*{Mister X}
Eine Person aus der Gruppe wird zu Mister X gewählt. Mister X bekommt 10 Minuten Vorsprung, danach versuchen die restlichen Teilnehmer in unterschiedlichen Teams diesen zu fangen.

\section*{Wikingerschach/Kubb}
Kubb ist ein rundenbasierter Wurfsport, wobei sich zwei Mannschaften an den Stirnseiten eines 5m x 8m großen Kubb-Feldes gegenüber stehen. Ein Team besteht aus 1 bis 6 Spielern und ein Kubb Spiel besteht aus 6 Wurfstäben, 10 Kubbs, 6 Begrenzungsstäben und einem König.

Zwei Teams werfen nacheinander mit den 6 Wurfhölzern möglichst viele gegnerische Basis-Kubbs ab. Nachdem alle Wurfhölzer geworfen wurden, ist das andere Team an der Reihe. Es sammelt die Wurfstäbe und gefallenen Kubbs ein. Aus einem Basis-Kubb wird nach dem Umfallen ein Feld-Kubb, denn nun muss das 2. Team versuchen, die eingesammelten Feld-Kubbs zurück in die Spielhälfte von Team 1 zu werfen und aufzustellen.

Team 2 muss jetzt mit den 6 Wurfhölzern die aufgestellten Feld-Kubbs umwerfen. Gelingt dies mit weniger als den 6 zur Verfügung stehenden Wurfstäben, so kann man versuchen, auf die gegnerischen Basis-Kubbs zu werfen, um weitere Kubbs ins Spiel zu bringen.

Werden diese nicht getroffen, passiert Folgendes: Das Team 1 darf nach dem erneuten Einwerfen der Kubbs von der Grundlinie bis zu dem ,,nicht gefallenen'' Feld-Kubb, nahe der Mittellinie vorgehen. Der Spieler kann dann auf einer imaginären Linie, die parallel zur Mittellinie verläuft, seinen Wurf ausführen.

Diejenige Mannschaft, der es mit ihren sechs Wurfstäben zuerst gelingt, alle Kubbs, die sich im gegnerischen Feld befinden, plus König umzuwerfen gewinnt das Kubb Spiel. Vorsicht ist geboten! Wird der König vorher durch einen Wurfstab oder Kubb zu Fall gebracht, gilt das Spiel als frühzeitig verloren. Wenn beim Königswurf ein gegnerischer Kubb (vorher nicht geräumter Kubb) getroffen wird, wird er an die gleiche Stelle zurückgestellt.

\section*{Beach Ball/Federball/Frisbee/Bumerang}
Diese Spielutensilien sollten in keinem Campingurlaub fehlen und ermöglichen dir auf der Wiese und am Strand aktiv zu sein.

\section*{Boule}
Nicht nur die Franzosen und Italiener lieben Boule, auch beim Camping kommt das Spiel gern zum Einsatz – schließlich gibt es auf Campingplätzen viel Platz und nicht selten auch einen Boule-Platz. Ziel ist es, seine Kugeln möglichst nahe an die kleine Zielkugel (Schweinchen) zu setzen und dabei die gegnerischen Kugeln vom Schweinchen wegzuschießen. Eine etwas leichtere und kinderfreundliche Variante ist das Crossboule, bei dem die Bälle den bekannten Hacky Sacks ähneln und somit überall gespielt werden können.

\section*{Räuber und Gendarm}
Kaum ein Kinderspiel ist beliebter als Räuber und Gendarm und in der Natur gibt es dafür reichlich Platz und Verstecke. Mindestens 4 Spieler sollten mitspielen. Dann werden ein Spielareal und zwei Gruppen, die Gendarmen und die Räuber, gebildet, die nicht unbedingt die gleiche Anzahl an Spielern haben müssen. In der Mitte des Areals wird ein Gefängnis festgelegt, in das die Gendarmen die gefangenen Räuber bringen. Die gefangenen Räuber können allerdings durch einfaches Abschlagen durch einen weiteren freien Räuber wieder befreit werden. Das Spiel ist vorbei, wenn alle Räuber gefangen wurden.

\end{document}
